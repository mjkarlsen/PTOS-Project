% Options for packages loaded elsewhere
\PassOptionsToPackage{unicode}{hyperref}
\PassOptionsToPackage{hyphens}{url}
%
\documentclass[
  ignorenonframetext,
]{beamer}
\usepackage{pgfpages}
\setbeamertemplate{caption}[numbered]
\setbeamertemplate{caption label separator}{: }
\setbeamercolor{caption name}{fg=normal text.fg}
\beamertemplatenavigationsymbolsempty
% Prevent slide breaks in the middle of a paragraph
\widowpenalties 1 10000
\raggedbottom
\setbeamertemplate{part page}{
  \centering
  \begin{beamercolorbox}[sep=16pt,center]{part title}
    \usebeamerfont{part title}\insertpart\par
  \end{beamercolorbox}
}
\setbeamertemplate{section page}{
  \centering
  \begin{beamercolorbox}[sep=12pt,center]{part title}
    \usebeamerfont{section title}\insertsection\par
  \end{beamercolorbox}
}
\setbeamertemplate{subsection page}{
  \centering
  \begin{beamercolorbox}[sep=8pt,center]{part title}
    \usebeamerfont{subsection title}\insertsubsection\par
  \end{beamercolorbox}
}
\AtBeginPart{
  \frame{\partpage}
}
\AtBeginSection{
  \ifbibliography
  \else
    \frame{\sectionpage}
  \fi
}
\AtBeginSubsection{
  \frame{\subsectionpage}
}
\usepackage{lmodern}
\usepackage{amssymb,amsmath}
\usepackage{ifxetex,ifluatex}
\ifnum 0\ifxetex 1\fi\ifluatex 1\fi=0 % if pdftex
  \usepackage[T1]{fontenc}
  \usepackage[utf8]{inputenc}
  \usepackage{textcomp} % provide euro and other symbols
\else % if luatex or xetex
  \usepackage{unicode-math}
  \defaultfontfeatures{Scale=MatchLowercase}
  \defaultfontfeatures[\rmfamily]{Ligatures=TeX,Scale=1}
\fi
% Use upquote if available, for straight quotes in verbatim environments
\IfFileExists{upquote.sty}{\usepackage{upquote}}{}
\IfFileExists{microtype.sty}{% use microtype if available
  \usepackage[]{microtype}
  \UseMicrotypeSet[protrusion]{basicmath} % disable protrusion for tt fonts
}{}
\makeatletter
\@ifundefined{KOMAClassName}{% if non-KOMA class
  \IfFileExists{parskip.sty}{%
    \usepackage{parskip}
  }{% else
    \setlength{\parindent}{0pt}
    \setlength{\parskip}{6pt plus 2pt minus 1pt}}
}{% if KOMA class
  \KOMAoptions{parskip=half}}
\makeatother
\usepackage{xcolor}
\IfFileExists{xurl.sty}{\usepackage{xurl}}{} % add URL line breaks if available
\IfFileExists{bookmark.sty}{\usepackage{bookmark}}{\usepackage{hyperref}}
\hypersetup{
  pdftitle={Compartment Syndrome},
  pdfauthor={Matt Carlson},
  hidelinks,
  pdfcreator={LaTeX via pandoc}}
\urlstyle{same} % disable monospaced font for URLs
\newif\ifbibliography
\usepackage{graphicx,grffile}
\makeatletter
\def\maxwidth{\ifdim\Gin@nat@width>\linewidth\linewidth\else\Gin@nat@width\fi}
\def\maxheight{\ifdim\Gin@nat@height>\textheight\textheight\else\Gin@nat@height\fi}
\makeatother
% Scale images if necessary, so that they will not overflow the page
% margins by default, and it is still possible to overwrite the defaults
% using explicit options in \includegraphics[width, height, ...]{}
\setkeys{Gin}{width=\maxwidth,height=\maxheight,keepaspectratio}
% Set default figure placement to htbp
\makeatletter
\def\fps@figure{htbp}
\makeatother
\setlength{\emergencystretch}{3em} % prevent overfull lines
\providecommand{\tightlist}{%
  \setlength{\itemsep}{0pt}\setlength{\parskip}{0pt}}
\setcounter{secnumdepth}{-\maxdimen} % remove section numbering

\title{Compartment Syndrome}
\subtitle{In Pediatric Patients After Forearm Fracture Treatment}
\author{Matt Carlson}
\date{June 7, 2020}

\begin{document}
\frame{\titlepage}

\begin{frame}{ Overview of Compartment Syndrome }
\protect\hypertarget{overview-of-compartment-syndrome}{}

\begin{block}{What is Compartment Syndrome?}

\end{block}

\begin{block}{Why does Compartment Syndrome occur?}

\end{block}

\begin{block}{How is Compartment Syndrome treated?}

\end{block}

\begin{block}{Review of Compartment Syndrome \& Fasciotomy}

\includegraphics[width=10.41667in,height=\textheight]{images/process_flow.png}

In this study, we will use Fasciotomy as the indicator for Compartment
Syndrome.

\begin{itemize}
\tightlist
\item
  We did this for the following reasons:

  \begin{itemize}
  \tightlist
  \item
    Compartment Syndrome is a diagnosis that is subject to human error

    \begin{itemize}
    \tightlist
    \item
      The PTOS results are sparse in terms of date, time, and accuracy
      of diagnosis
    \end{itemize}
  \item
    Fasciotomy is a procedure that takes place because of Compartment
    Syndrome

    \begin{itemize}
    \tightlist
    \item
      This is a surgical procedure and it is well documented in the PTOS
      data
    \end{itemize}
  \end{itemize}
\end{itemize}

\end{block}

\end{frame}

\begin{frame}{ Overview of Forearm Factures }
\protect\hypertarget{overview-of-forearm-factures}{}

\begin{block}{Four Types of Forearm Fracture Treatments}

\begin{enumerate}
\tightlist
\item
  Open Reduction with Internal Fixation

  \begin{itemize}
  \tightlist
  \item
    Most common and preferred method for forearm fracture treatment
  \end{itemize}
\item
  Closed Reduction without Internal Fixation

  \begin{itemize}
  \tightlist
  \item
    Involves manually setting the bone back into place and use of hard
    cast for stability
  \end{itemize}
\item
  Closed Reduction with Internal Fixation

  \begin{itemize}
  \tightlist
  \item
    Not a surgical procedure, minor incision in forearm using pins for
    stability
  \end{itemize}
\item
  Open Reduction without Internal Fixation

  \begin{itemize}
  \tightlist
  \item
    Surgical procedure, only performed if there is no need for internal
    fixator
  \end{itemize}
\end{enumerate}

\end{block}

\begin{block}{Review of Open Reduction Forearm Fractures}

\end{block}

\begin{block}{Review of Closed Reduction Forearm Fractures}

\end{block}

\begin{block}{Review of the Four Forearm Fracture Treatments}

\includegraphics[width=\textwidth,height=2.86458in]{images/forearm_fractures.png}

\begin{itemize}
\tightlist
\item
  These are surgical procedures
\item
  More traumatic fractures requiring invasive techniques
\item
  Use of plates, screws, and external cast to secure fracture site
\item
  Patients are sedated using anesthesia
\end{itemize}

\begin{itemize}
\tightlist
\item
  These are not surgical procedures
\item
  Technique performed by manipulation for less traumatic fractures
\item
  Use of pins and cast to hold bones in place for healing
\item
  Painful technique for patient and performed with or without general
  anesthetic
\end{itemize}

\end{block}

\end{frame}

\begin{frame}{ Pennsylvania Trauma Outcome System }
\protect\hypertarget{pennsylvania-trauma-outcome-system}{}

\begin{block}{What is Pennsylvania Trauma Outcome System?}

\includegraphics[width=2.60417in,height=\textheight]{images/ptos.png}

This is the primary data source used for our analysis of Compartment
Syndrome in Pediatrics Patients.

\begin{itemize}
\tightlist
\item
  Approximately 530k anonymized patient records and 1,400 columns of
  descriptive information regarding each procedure
\item
  Medical results occurring between 2010 - 2015
\item
  Patient demographics, diagnosis, prehospital procedures, surgical
  procedures, and outcomes
\item
  Uses International Statistical Classification of Diseases and Related
  Health Problems (ICD) medical coding to classify causes of injury,
  examination results, processes, and treatments.
\end{itemize}

Pennsylvania Trauma Outcome System will be referred to as PTOS going
forward.

\end{block}

\end{frame}

\begin{frame}{ Objective of Compartment Syndrome Analysis }
\protect\hypertarget{objective-of-compartment-syndrome-analysis}{}

\begin{block}{Objective of Analysis}

Identify the likelihood of compartment syndrome in pediatric patients
who suffered from forearm fractures .

\includegraphics[width=10.41667in,height=\textheight]{images/objective.png}

\end{block}

\end{frame}

\begin{frame}{Data Preparation}
\protect\hypertarget{data-preparation}{}

\begin{block}{Data Preparation}

\begin{itemize}
\tightlist
\item
  Developed a custom R package, called traumaR to translate all medical
  codes into human friendly terms

  \begin{itemize}
  \tightlist
  \item
    Wrote over 50 functions to automate this work and make it repeatable
  \end{itemize}
\item
  PTOS Data consists of 530k anonymized patient records and 1,400
  columns

  \begin{itemize}
  \tightlist
  \item
    Columns mostly consisting of medical codes which required
    translation
  \item
    Necessary to normalize the data into structured data frame
  \end{itemize}
\item
  The final results create two main data frames joined together by
  patient\_id

  \begin{itemize}
  \tightlist
  \item
    Patient Information
  \item
    Medical Procedures
  \end{itemize}
\end{itemize}

\href{https://github.com/mjkarlsen/traumaR}{traumaR}

\end{block}

\end{frame}

\begin{frame}{Exploratory Data Analysis}
\protect\hypertarget{exploratory-data-analysis}{}

\begin{block}{EDA: Patient}

\end{block}

\begin{block}{EDA: Forearm Fractures}

\end{block}

\begin{block}{EDA: Fasciotomy}

\end{block}

\begin{block}{EDA: Fasciotomy Results}

\includegraphics[width=10.41667in,height=\textheight]{images/gt_graph_1.png}

The Open Reduction with Internal Fixation is the most common technique
across the medical field

\end{block}

\begin{block}{EDA: Fasciotomy Results}

\includegraphics[width=6.77083in,height=\textheight]{images/gt_graph_2.png}

\begin{itemize}
\tightlist
\item
  Pediatric patients tend to be treated with Closed Reduction and
  casting

  \begin{itemize}
  \tightlist
  \item
    Most of the injuries result from falling on the playground or at
    home
  \item
    These fractures are usually less traumatic and do not require
    invasive techniques
  \end{itemize}
\end{itemize}

\end{block}

\begin{block}{Association Analysis of Medical Procedures}

\begin{itemize}
\tightlist
\item
  The red lines represent common paths in the series of procedures that
  led to a fasciotomy.
\item
  Removed CAT scans (87.03, 87.41, 87.71, 88.01, 88.38), and suture code
  (83.65) procedures as they made up a majority of the relationships.

  \begin{itemize}
  \tightlist
  \item
    The number of CAT scans were related to the severity of the injury
    and suturing is just part of the everyday surgery.
  \end{itemize}
\end{itemize}

\end{block}

\end{frame}

\begin{frame}{Model Development}
\protect\hypertarget{model-development}{}

\begin{block}{Predicting Fasciotomy}

\includegraphics[width=9.375in,height=\textheight]{images/model_building.png}

\end{block}

\begin{block}{Model Pre-Processing}

\includegraphics[width=7.8125in,height=\textheight]{images/class_imbalance.png}

\begin{itemize}
\tightlist
\item
  Up-sampled the minority class in training data

  \begin{itemize}
  \tightlist
  \item
    Fasciotomy == TRUE
  \end{itemize}
\item
  Use F-Measure as measure of model accuracy

  \begin{itemize}
  \tightlist
  \item
    F-Measure combines precision and recall into a single score
  \item
    Highly recommended for imbalanced classifications
  \end{itemize}
\end{itemize}

\end{block}

\begin{block}{Feature Engineering}

\begin{itemize}
\tightlist
\item
  Step Other

  \begin{itemize}
  \tightlist
  \item
    Groups categorical information into `other' category whose
    individual levels do not exceed preset threshold.
  \end{itemize}
\item
  Step Dummy

  \begin{itemize}
  \tightlist
  \item
    Converts character or factors into dummy variables
  \end{itemize}
\item
  Step Zero Variance

  \begin{itemize}
  \tightlist
  \item
    Removes variables that contain only a single value

    \begin{itemize}
    \tightlist
    \item
      This is a precautionary step to ensure that Step Other captured
      all low occurrence into `Other'
    \end{itemize}
  \end{itemize}
\item
  Step Omit NA

  \begin{itemize}
  \tightlist
  \item
    Remove any observations with missing values
  \item
    Some models cannot handle missing values
  \end{itemize}
\end{itemize}

\end{block}

\begin{block}{Models Deployed}

\includegraphics[width=10.41667in,height=\textheight]{images/model_specs.png}

\end{block}

\begin{block}{Model Results: In-Sample Accuracy}

\includegraphics[width=7.8125in,height=\textheight]{images/gt_train_graph.png}

\end{block}

\begin{block}{Model Results: Out-of-Sample Accuracy}

\includegraphics[width=7.8125in,height=\textheight]{images/gt_test_graph.png}

\end{block}

\begin{block}{RandomForest Variable Importance}

\includegraphics[width=\textwidth,height=5.20833in]{images/rf_vip.png}

Variable Importance

\begin{itemize}
\tightlist
\item
  Specific Types of Injuries

  \begin{itemize}
  \tightlist
  \item
    Fall from Stairs, Motor Vehicles, Motorcyclist, Firearm Explosives
  \end{itemize}
\item
  Location of Injury

  \begin{itemize}
  \tightlist
  \item
    Home, Street/Highway
  \end{itemize}
\item
  Patient Demographics

  \begin{itemize}
  \tightlist
  \item
    Male, White, and Age 22
  \end{itemize}
\item
  Medical Procedure

  \begin{itemize}
  \tightlist
  \item
    Operations on Skin and Subcutaneous Tissue
  \end{itemize}
\end{itemize}

\end{block}

\begin{block}{ROC Curve}

Train Data

Test Data

\end{block}

\begin{block}{Area Under Curve Scores}

\end{block}

\end{frame}

\begin{frame}{Visualization and Application}
\protect\hypertarget{visualization-and-application}{}

\begin{block}{Shiny Dashboard: Web Version}

\end{block}

\begin{block}{Shiny Dashboard: Phone Application}

\end{block}

\end{frame}

\begin{frame}{Review of Deliverables \& Output}
\protect\hypertarget{review-of-deliverables-output}{}

\begin{block}{Review of Deliverables \& Output}

\begin{itemize}
\tightlist
\item
  Created a model that can predict the likelihood of a patient
  developing compartment syndrome based on patient demographics and
  medical history.

  \begin{itemize}
  \tightlist
  \item
    Overall, the results showed that severity of injury and cause of
    injury were the leading predictors of compartment syndrome.
  \item
    There was little information presented to confirmed that the order
    of medical procedures had any part of increasing compartment
    syndrome.
  \end{itemize}
\item
  Developed a shiny application to assess likelihood of compartment
  syndrome with user-friendly interactive interface.
\item
  Built a customer R Package, traumaR, and published it to GitHub and
  made it publicly accessible.

  \begin{itemize}
  \tightlist
  \item
    The PTOS data provides significant insights for the medical field;
    however its initial data structure invites challenges for
    individuals lacking data manipulation expertise.
  \end{itemize}
\item
  Contributed to another R Package, tidytable, during the project which
  is published on CRAN.
\end{itemize}

\end{block}

\end{frame}

\begin{frame}{Thank you!}
\protect\hypertarget{thank-you}{}

This presentation was created in R using RMarkdown and ioSlides,
\href{https://github.com/mjkarlsen/PTOS-Project}{Compartment Syndrome
Project}.

\end{frame}

\end{document}
